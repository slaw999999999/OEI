\documentclass{article}

\usepackage{polski}
\usepackage[utf8]{inputenc}
\pagestyle{empty}
\usepackage{geometry}
\geometry{left=2cm,right=2cm,top=2cm,bottom=2cm}
\usepackage{titlesec}
\titlelabel{\thetitle.\ }
\usepackage{multirow}
\usepackage{lastpage}
\usepackage{amsmath, amsthm,amsfonts,amssymb,enumerate,bm,titlesec,amsthm}
\renewcommand*{\proofname}{\textbf{\textup{Proof.}}}
\linespread{1.5}

\usepackage{hyperref}

\title{Praca zaliczeniowa z oceny efektywności inwestycji}
\author{Sławomir Jasina}
\date{30.05.2022r.}
\begin{document}
	\maketitle
	\newpage
	\tableofcontents
	\newpage
	
	\href{https://github.com/slaw999999999/OEI/blob/main/README.md}{Kompilacja kodu online}
	
	
	\vspace*{1,1cm} 
	\phantomsection
	\addcontentsline{toc}{section}{Zadanie 1}
	\section*{Zadanie 1}
	

	\subsection*{Rozwiązanie zadania:}
	
	
	\subsection*{Rozwiązanie zadania w Python:}
	\href{https://github.com/slaw999999999/OEI/blob/main/Zadanie1.ipynb}{Rozwiązanie zadania 1 Python}
	
	Okres zwrotu to okres, po którym zysk z poniesionej inwestycji pokrył poniesione koszty. Innymi słowy: Jest to okres, po któym inwestycja zaczyna "na siebie zarabiać". \newline
	
	
	
	W zadania do ocenie zostały poddane 3 inwestycje.	
	 Najkorzystniejszą okazała się być inwestycja pierwsza, która zwróciła się po 8 latach. Drugą najkorzystniejszą okzała się być inwestycja druga zwracja się po 9 latach, zaś trzecia najmniej najkorzystna z okresem zwrotu aż 10 lat.
	
	
	
	\newpage
	\addcontentsline{toc}{section}{Zadanie 2}
	\section*{Zadanie 2}
	
	
	\subsection*{Rozwiązanie zadania:}
	
	
	\subsection*{Rozwiązanie zadania w Python:}
	\href{https://github.com/slaw999999999/OEI/blob/main/Zadanie1.ipynb}{Rozwiązanie zadania 2 Python}
	
	Celem zadania było dokonanie wyboru najbardziej opłacalnego projektu spośród 4 projektów: P1, P2, P3 P4, przy stopie dyskontowej $15\%$. \newline
	
	\textbf{Wartości bieżącej netto (NPV) jest to suma przepływów pieniężnych netto zdyskontowanych
	oddzielnie dla każdego okresu, wygenerowanych w całym ekonomicznym cyklu życia rozważanej
	inwestycji, przy stałym poziomie stopy dyskontowej.}

	Po wykonaniu niezbędnych obliczeń ranking najkorzystniejszych inwestycji przedstawił się nastepująco:
	\begin{itemize}
		\item 
		P3 - NPV = 7369.30 
		\item 
		P1 - NPV = 7073.08
		\item
		P2 - NPV = 6962.25
		\item 
		P4 - NPV = 5841.19
		
	\end{itemize}
	
	Widzimy, że najkorzystniejszą inwestycją okazuje się być inwestycja numer 3.
	\newpage
	\addcontentsline{toc}{section}{Zadanie 3}
	\section*{Zadanie 3}
	
	
	\subsection*{Rozwiązanie zadania:}
	
	
	\subsection*{Rozwiązanie zadania w Python:}
	\href{https://github.com/slaw999999999/OEI/blob/main/Zadanie1.ipynb}{Rozwiązanie zadania 3 Python}
	
	Celem zadania było wyznaczenie wewnętrzenj stopy zwrotu inwestycji.\newline
	
	\textbf{Wewnętrzna stopa zwrotu to stopa dyskontowa, dla której
		zachodzi równość pomiędzy wartością zaktualizowaną wydatków pieniężnych a wartością zaktualizowaną
		wpływów pieniężnych. Jest to więc taka stopa procentowa, dla której wartość bieżąca netto ocenianej
		inwestycji jest równa zero.}
	
	Schemat rozwiązania był następujący. Pierwsze zostały wyznaczone 2 stopy procentowe, dla których NPV inwestycji jest bbliskie 0. Przy czym dla pierwszej NPV ujemne, a dla drugiej dodatnie.\newline
	
	Po wyznaczeniu odpowiednich $r_1$ i $r_2$ ostateczny rezultat został obliczony ze wzrou na IRR:
	
	$$ IRR = R1 + \frac{NPV_1\cdot(R2-R1)}{NPV_1-NPV_2}= 0.0904 $$ 
	
	\newpage
	\addcontentsline{toc}{section}{Zadanie 4}
	\section*{Zadanie 4}
	
	
	\subsection*{Rozwiązanie zadania:}
	
	
	\subsection*{Rozwiązanie zadania w Python:}
	\href{https://github.com/slaw999999999/OEI/blob/main/Zadanie1.ipynb}{Rozwiązanie zadania 4 Python}
	
	W zadaniu zostały przedstawione 3 projekty inwestycyjne. Cel zadania skupiał się naocenie które z projektów powinny być przyjęte do realizacji, przy założeniach, że:
	
	\begin{itemize}
		\item
		 r= $14\%$
		\item
		 Firma może zainwestować 1000zł.	
	\end{itemize}
	
	
	Analiza opłącalności iwnestycji została przeprowadzona ponownie za pomocą NPV, z tą różnicą, że na końcu został odjęty wkłąd pieniężny firmy w inwestycję.
	
	Po wykonaniu niezbędnych obliczeń ranking najkorzystniejszych inwestycji przedstawił się nastepująco:
	\begin{itemize}
		\item 
		P3 - NPV = 675.2373
		\item 
		P2 - NPV = -214.3390
		\item
		P1 - NPV = -745.9295
	\end{itemize}

	Zwróćmy uwagę na to, że jedyną inwestycją z dodatnią wartością NPV jest inwestycja numer 1. Oznacza to, że jako jedyna jest inwestycją korzystną. Pozostałe mają wartości ujemne, więc należy  je odrzucić.
	\newpage
	\addcontentsline{toc}{section}{Zadanie 5}
	\section*{Zadanie 5}
	
	
	\subsection*{Rozwiązanie zadania:}
	
	
	\subsection*{Rozwiązanie zadania w Python:}
	\href{https://github.com/slaw999999999/OEI/blob/main/Zadanie1.ipynb}{Rozwiązanie zadania 5 Python}
	
	Celem zadania było ocenic opłacalność 2 inwestycji za pomocą metody MIRR, przy założeniach:
	
	\begin{itemize}
		\item 
		stopa
		dyskontowa = 
		$14 \%$,
		
		\item 
		stopa
		reinwestycji = 
		$8 \%$
	\end{itemize}

	\textbf{MIRR (ang. Modified Internal Rate of Return) - czyli zmodyfikowana wewnętrzna stopa zwrotu, to dynamiczna metoda oceny efektywności ekonomicznej projektów inwestycyjnych, a także wskaźnik finansowy, wyznaczony w oparciu o tę metodę. Uwzględnia ona zmiany wartości pieniądza w czasie i jest oparta o analizę zdyskontowanych przepływów pieniężnych.}\newline 
	
	Pierwsze należało policzyć sumę wszystkich dodatnich i ujemnych przepływów pieniężnych. Dla dodatnich przepływow pieniężnych kolejne sumy były dodawane z uwzględnieniem stopy reinwestycji, zaś ujemnych stopy dyskontowej. Ostatecznie przy użyciu wzoru:
	
		$$ MIRR_1 = \left(\frac{PLUS_1}{-MINUS_1}\right)^{\frac{1}{n}}-1 = 0.0746 $$
		
		$$ MIRR_2 = \left(\frac{PLUS_2}{-MINUS_2}\right)^{\frac{1}{n}}-1 = 0.0701 $$
	
	Otrzymujemy, że bardziej korzystną inwestycją jest inwestycja pierwsza.
	\newpage
	\addcontentsline{toc}{section}{Zadanie 6}
	\section*{Zadanie 6}
	
	
	\subsection*{Rozwiązanie zadania:}
	
	
	\subsection*{Rozwiązanie zadania w Python:}
	\href{https://github.com/slaw999999999/OEI/blob/main/Zadanie1.ipynb}{Rozwiązanie zadania 6 Python}
	
	
	 Cel zadania skupiał się wokół wyznaczenia ilościowego progu rentowności \newline
	 	 
	 \textbf{Wskaźnik rentowności (PI – ang. Profitability Index) jest to iloraz sumy zdyskontowanych dodatnich
	 	przepływów pieniężnych netto do wartości bezwzględnej sumy zdyskontowanych ujemnych przepływów
	 	pieniężnych netto.}\newline
 	
  	Po wykonaniu niezbędnych rachunków we wzorze:
  	
  		$$Rentownosc = \frac{KosztyProd}{CenaJedn-JednKosztZmien} $$
  		
  	Okazało się, że ilościowa rentowność wynosi 148 sztuk. Oznacza to, że firma musi wyprodukować minimalnie 148 sztuk, żeby nie byc stratną na produkcji.
 	
 	
	\newpage
	\addcontentsline{toc}{section}{Zadanie 7}
	\section*{Zadanie 7}
	
	
	\subsection*{Rozwiązanie zadania:}
	
	
	\subsection*{Rozwiązanie zadania w Python:}
	\href{https://github.com/slaw999999999/OEI/blob/main/Zadanie1.ipynb}{Rozwiązanie zadania 7 Python}
	
	Celem zadania było ocenić opłacalność dwóch inwestycji za pomocą NPV przy podejściu probabilistycznym. Założenia:
	\begin{itemize}
		\item 
		stopa
		dyskontowa = 
		$11 \%$,
		
		\item 
		nakładów inwestycyjnych =  5710
	\end{itemize}
		
		Pierwsze niezbędne było wyznaczenie wartości oczekiwanych dla dwóch iwnestycji. Następnie ich wariancji.
		
		Posiadając wartości wyżej wymienionych parametrów dla obydwu inwestycji oszacowane zostały wartości probabilistyczne NPV:
		
		\begin{itemize}
			\item 
			P1 - NPV probabilistyczne: 146.3837	
			\item 
			P2 - NPV probabilistyczne wynosi: -275.7426
		\end{itemize}
			
		Okazało się, że najkorzystniejszą inwestcją jest inwestycja pierwsza
	
	\newpage
	\addcontentsline{toc}{section}{Zadanie 8}
	\section*{Zadanie 8}
	
	
	\subsection*{Rozwiązanie zadania:}
	
	
	\subsection*{Rozwiązanie zadania w Python:}
	\href{https://github.com/slaw999999999/OEI/blob/main/Zadanie1.ipynb}{Rozwiązanie zadania 8 Python}
	
	Opłacalność inwestcji została oszacowana przy pomocy trzech różnych sposobów.
	\begin{itemize}
		\item 
		Pierwszy sposób to ocena po ilu latach iwnestycja się zwróci o raz czy wogóle się zwróci. Przy początkowej inwestycji okazało się, że inwestycja sie nie zwróci i jest nieopłacalna, zaś przy nakładzie 10 000 inwestycja zacznie być opłącalna dopiero po 14 latach.	
		\item 
		Drugim sposobem było użycie NPV. Przy stopie dyskontowej wynoszącej $5\%$ inwestycja okazała się być opłacalną. Całość przebiegu inwestycji została zwizualizowana graficznie.
	\end{itemize}

\end{document}